\documentclass[8pt,notheorems]{beamer}
\usetheme{Copenhagen}
\usepackage[utf8]{inputenc}
\usepackage[T1]{fontenc}
\usepackage{beamerthemesplit}
\usepackage{graphicx}
\usepackage{tkz-graph}
\usepackage{color}
\usepackage{listings}

\usepackage{amsmath,amsfonts,amsthm,t1enc}
\usepackage{fourier}
\usepackage{listings}
\usepackage{amsmath}
\usepackage{tikz}
\usepackage{booktabs}
\newcommand{\ra}[1]{\renewcommand{\arraystretch}{#1}}
\usetikzlibrary{positioning}
\usetikzlibrary{fit}
\usetikzlibrary{backgrounds}
\usetikzlibrary{calc}
\usetikzlibrary{shapes}
\usetikzlibrary{mindmap}
\usetikzlibrary{decorations.text}

\usetikzlibrary{automata,arrows,positioning,calc}
\setbeamertemplate{footline}{\hfill \insertframenumber/\inserttotalframenumber}
\def \si {\sigma}
\def \la {\lambda}
\def \al {\alpha}
% \def\e*{\end{eqnarray*}}
\def \di{\displaystyle}

\def \E{\mathbb E}
\def \N{\mathbb N}
\def \Z{\mathbb Z}
\def \NZ{\mathbb{N}_0}
\def \I{\mathbb I}
\def \w{\widehat}
\def \P {\mathbb P}
\def \V{\mathbb V}


\newcommand{\CL}{\mathbb{C}}
\newcommand{\R}{\mathbb{R}}
\newcommand{\nat}{{\mathbb N}}
\newcommand{\Laplace}{\mathscr{L}}
\newcommand{\e}{\mathrm{e}}
\newcommand{\ve}{\bm{\mathrm{e}}} % vector e

\renewcommand{\L}{\mathcal{L}} % e.g. L^2 loss.

\newcommand{\ih}{\mathrm{i}}
\newcommand{\oh}{{\mathrm{o}}}
\newcommand{\Oh}{{\mathcal{O}}}
\newcommand{\Exp}{\mathbb{E}}
\newcommand{\va}{\textbf{v.a.}}
\newcommand{\iid}{\textbf{i.i.d.}}

\newcommand{\Norm}{\mathcal{N}}
\newcommand{\LN}{\mathcal{LN}}
\newcommand{\SLN}{\mathcal{SLN}}

\renewcommand{\Pr}{\mathbb{P}}
\newcommand{\Ind}{\mathbb I}
\newcommand\bfsigma{\bm{\sigma}}
\newcommand\bfSigma{\bm{\Sigma}}
\newcommand\bfLambda{\bm{\Lambda}}
\newcommand{\stimes}{{\times}}
\def \limsup{\underset{n\rightarrow+\infty}{\overline{\lim}}}
\def \liminf{\underset{n\rightarrow+\infty}{\underline{\lim}}}

\setbeamertemplate{theorem}[ams style]
\setbeamertemplate{theorems}[numbered]
%\makeatletter
%\def\th@mystyle{%
%    \normalfont % body font
%    \setbeamercolor{block title example}{bg=orange,fg=white}
%    \setbeamercolor{block body example}{bg=blue!20,fg=black}
%    \def\inserttheoremblockenv{block}
%  }
%\makeatother
%\theoremstyle{mystyle}

\makeatletter
    \ifbeamer@countsect
      \newtheorem{theorem}{\translate{Theorem}}[section]
    \else
      \newtheorem{theorem}{\translate{Theorem}}
    \fi
    \newtheorem{corollary}{\translate{Corollary}}
    \newtheorem{prop}{\translate{Proposition}}
    \newtheorem{lemma}{\translate{Lemma}}
    \newtheorem{problem}{\translate{Problem}}
    \newtheorem{remark}{\translate{Remark}}
    \newtheorem{solution}{\translate{Solution}}

    \theoremstyle{definition}
    \newtheorem{definition}{\translate{Definition}}
    \newtheorem{definitions}{\translate{Definitions}}

    \theoremstyle{example}
    \newtheorem{example}{\translate{Example}}
    \newtheorem{examples}{\translate{Examples}}

\makeatletter
\def\th@mystyle{%
    \normalfont % body font
    \setbeamercolor{block title example}{bg=orange,fg=white}
    \setbeamercolor{block body example}{bg=orange!20,fg=black}
    \def\inserttheoremblockenv{exampleblock}
  }
\makeatother
\theoremstyle{mystyle}
\newtheorem{fact}{Fact}






    % Compatibility
    \newtheorem{Beispiel}{Beispiel}
    \newtheorem{Beispiele}{Beispiele}
    \theoremstyle{plain}
    \newtheorem{Loesung}{L\"osung}
    \newtheorem{Satz}{Satz}
    \newtheorem{Folgerung}{Folgerung}
    \newtheorem{Fakt}{Fakt}
    \newenvironment{Beweis}{\begin{proof}[Beweis.]}{\end{proof}}
    \newenvironment{Lemma}{\begin{lemma}}{\end{lemma}}
    \newenvironment{Proof}{\begin{proof}}{\end{proof}}
    \newenvironment{Theorem}{\begin{theorem}}{\end{theorem}}
    \newenvironment{Problem}{\begin{problem}}{\end{problem}}
    \newenvironment{Corollary}{\begin{corollary}}{\end{corollary}}
    \newenvironment{Example}{\begin{example}}{\end{example}}
    \newenvironment{Examples}{\begin{examples}}{\end{examples}}
    \newenvironment{Definition}{\begin{definition}}{\end{definition}}
\makeatother











% ============================================================
% Title
% ============================================================

\title[]{Modélisation Charge Sinistre M2 Actuariat}
\subtitle{Chapitre II: Modèle individuel}
\author{Pierre-Olivier Goffard}
\institute{
	   Université de Lyon 1\\
	ISFA\\
	   \texttt{pierre-olivier.goffard@univ-lyon1.fr}
	  }
\date{
ISFA\\
\today}
\lstset{language=SAS}
\begin{document}

\frame{\titlepage}


% ============================================================
\section{Définition de $X_{\text{ind}}$}
\begin{frame}[allowframebreaks]
\underline{I. Définition de $X_{\text{ind}}$}\\
Le modèle individuel consiste à modéliser la charge totale de sinsitre sur une période en sommant les montants de sinistres pour chaque contrat. On définit ainsi la variable aléatoire 
\begin{equation}\label{eq:modele_individuel}
X_{\text{ind}} = \sum_{i = 1}^{n}I_iW_i,
\end{equation}
où
\begin{itemize}
    \item $n$ correspond au nombre de contrats en portefeuille
    \item $I_i\sim\text{Ber}(p_i)$, de loi 
    $$
    \P(I_i = 1) = p_i\text{ et }\P(I_i = 1) = 1-p_i = q_i
    $$
    indique si un sinistre a été reporté pour le contrat $i$ ou non.
    \item $W_i$ est une variable aléatoire positive égale au cumul des indemnisations générées par le contrat $i$.
\end{itemize}
Ce modèle permet de tenir compte de l'hétérogénéité du portefeuille et garantit une mesure très précise du risque. En effet, la probabilité d'observer un sinistre diffère pour chaque contrat, de même que la loi du montants des indemnisation. Deux problèmes se pose néanmoins
\begin{enumerate}
    \item La loi de $X_{ind}$ est inaccessible sauf par simulation
    \item La calibration de modèle paramétriques pour les différentes variables aléatoires nécessite un historique de données important (voir prohibitif) et ce pour chaque contrat.
\end{enumerate}
\end{frame}
\section{Moments de $X_{\text{ind}}$}
\begin{frame}[allowframebreaks]
\underline{II. Moments de $X_{\text{ind}}$}\\
Les moments et la fonction génératrices des moments de $X_{\text{ind}}$ peuvent s'exprimer en fonction de $p_1,\ldots,p_n$ et de l'espérance de $W_1,\ldots, W_n$ à condition d'imposer une condition d'indépendance entre les contrats. 
\begin{prop}[Variance et fonction génératrice des moments]
Posons $q_i = 1-p_i$ et supposons que $I_1,\ldots,I_n,W_1,\ldots, W_n$ soient des \va indépendantes, alors 
\begin{enumerate}
    \item L'espérance de $X_{\text{ind}}$ est donnée par 
    $$
    \E(X_{\text{ind}}) =  \sum_{i = 1}^{n}p_i\E(W_i).
    $$
    \item La variance de $X_{\text{ind}}$ est donnée par
    \begin{equation}\label{eq:var_modele_individuel}
    \V(X_{\text{ind}}) =  \sum_{i = 1}^{n}\left[p_iq_i\E(W_i)^2 + p_i\V(W_i)\right].
    \end{equation}
    \item La fonction génératrice des moments de $X_{\text{ind}}$ est donnée par
    $$
    m_{X_{\text{ind}}}(s) =\E\left(e^{sX_{\text{ind}}}\right) = \prod_{i = 1}^{n}\left[p_im_{W_i}(s) + q_i\right],\text{ pour }s\in \R.
    $$
\end{enumerate}
\end{prop}
\underline{preuve:}\\
\begin{enumerate}
    \item
    $$
    \E(X_{\text{Ind}}) = \sum_{i=1}^n\E(I_iW_i)=\sum_{i=1}^n\E(I_i)\E(W_i) =\sum_{i=1}^np_i\E(W_i)   
    $$
    \item
    \begin{equation}\label{eq:master_variance}
    \V(X_{\text{Ind}}) = \sum_{i=1}^n\V(I_iW_i)=\sum_{i=1}^n\left\{\V[\E(I_iW_i|I_i)]+\E[\V(I_iW_i|I_i)]\right\}.   
    \end{equation}
    on a 
    \begin{equation}\label{eq:var_1}
    \V[\E(I_iW_i|I_i)]= \V[I_i\E(W_i|I_i)] = \V(I_i)\E(W_i)^2 = p_iq_i\E(W_i)^2
    \end{equation}
    d'une part, et 
    \begin{equation}\label{eq:var_2}
    \E[\V(I_iW_i|I_i)]= \V(W_iI_i|I_i = 1)\P(I_i= 1)+ \V(W_iI_i|I_i = 0)\P(I_i= 0) = p_i\V(W_i|I_i) = p_i\V(W_i)
    \end{equation}
    d'autre part. On obtient \eqref{eq:var_modele_individuel} en remplaçant \eqref{eq:var_1} et \eqref{eq:var_2} dans \eqref{eq:master_variance}. 
    \item Examen :)
\end{enumerate}
\end{frame}
\section{Distribution de $X_\text{ind}$}
\begin{frame}[allowframebreaks]
\underline{III. Distribution de $X_\text{ind}$}\\
L'étude de la distribution de $X_\text{ind}$ nécessite de faire la même hypothèse d'indépendance que précédemment. Il faut ensuite conditionné par rapport à toute les situations possibles par exemple que seul le premier contrat subisse un sinistre, ou que seuls le premier et le deuxième subissent une sinistre. Pour chaque cas il faut alors calculer des produits de convolution.\\

Par exemple si les trois premiers contrats sont sinistrés alors 
$$
X_\text{ind}\Big\rvert\bigcap_{k=1}^{3}{I_k=1}\cap\bigcap_{l=4}^{n}{I_l=0} = W_1+W_2+W_3 
$$
La densité de la somme de deux variables aléatoires indépendante et positives $W_1+W_2$ est donnée par 
$$
f_{W_1+W_2}(x) = (f_{W_1}\ast f_{W_2})(x)=\int_{0}^{x}f_{W_1}(y)f_{W_2}(x-y)\text{d}y
$$
puis 

$$
f_{W_1+W_2+W_3}(x) = ((f_{W_1}\ast f_{W_2})\ast f_{W_3})(x) = \int_0^x\int_{0}^{z}f_{W_1}(y)f_{W_2}(z-y) f_{W_3}(x-z)\text{d}y\text{d}z
$$
Les calculs deviennent rapidemment prohibitifs... De Pril \cite{depril_1989} proposa en 1989 une méthode de calcul récursive en faisant les deux approximation suivantes 
\begin{itemize}
    \item Regroupement des contrats en classe homogène $\left\{(i,j)\text{ , }i = 1,\ldots, a\text{ et }j = 1,\ldots,b\right\}$ contenant $n_{ij}$. Un contrat qui appartient à la classe $(i,j)$ 
    \begin{itemize}
        \item a une probabilité d'être sinistré $p_i$ 
        \item a un cumul d'indemnisation distribué comme $W_j$
    \end{itemize}
    \item Les \textbf{v.a.} $W_j,\text{ }j = 1,\ldots, b$ sont discrètes de loi de probabilité 
    $$
    p_{W_j}(w) = \P(W_j = w)\text{, }w = 1,\ldots, m_j\text{, et }j = 1,\ldots b,
    $$
    où $m_j$ correspond au montant maximal pour un contrat de la classe $j.$
\end{itemize}
On note que $n = \sum_{i = 1}^{a}\sum_{j = 1}^{b}n_{ij}$. 
\begin{table}[h!]\centering
\ra{1.3}
\scriptsize
\begin{tabular}{@{}ll|rrrrr@{}}\toprule
&&\multicolumn{5}{c}{Classe montant des sinistres}\\
\cmidrule{3-7}
 \multicolumn{2}{c|}{Classe occurence sinistre}&$1$&$\ldots$&j&$\ldots$&$b$\\
\midrule
 & 1&$n_{11}$ &  & $n_{1j}$ & &$n_{1b}$  \\
& 2&$n_{21}$ &  & $n_{2j}$ & &$n_{2b}$  \\
&$\vdots$& &  &  & &  \\
& i&$n_{i1}$ &  & $n_{ij}$ & &$n_{ib}$  \\
 &$\vdots$ & &  &  & &  \\
 &a&$n_{a1}$ &  & $n_{aj}$ & &$n_{ab}$  \\
\bottomrule
\end{tabular}
\end{table}
La charge totale de sinistre est une variable aléatoire discrète dont La loi de probabilité 
$$
p_{X_{\text{ind}}}(x) = \P(X_{\text{ind}} = x), x = 0,1,\ldots, m, \text{ où }m = \sum_{i = 1}^a\sum_{j = 1}^b n_{ij}m_i,
$$
peut-être évaluée via la formule récursive suivante
\begin{theorem}[Recurrence de De Pril]
La loi de probabilité de $X_{\text{Ind}}$ est donnée par
$$
p_{X_{\text{ind}}}(0) = \prod_{i = 1}^a (q_i)^{n_i}
$$
où $n_i = \sum_{j = 1}^b n_{ij}$ et 
$$
x.p_{X_{ind}}(x) = \sum_{i = 1}^{a}\sum_{j = 1}^{b}n_{ij}v_{ij}(x),
$$
avec 
$$
v_{ij}(x) = \frac{p_i}{q_i}\sum_{l = 1}^{x\land m_j}p_{W_{j}}(l)\left[lp_{X_{\text{Ind}}}(x-l)-v_{ij}(x-l)\right],\text{ }x = 1,\ldots m,\text{ et }v_{ij}(0) = 0.
$$

\end{theorem}
\underline{preuve:} En TD!\\
La fonction génératrice des probabilité de $X_{\text{Ind}}$ est donnée par 
\begin{equation}\label{eq:fgp_X_ind}
G_{X_{\text{Ind}}}(s)=\prod_{i = 1}^{a}\prod_{i = 1}^{b}\left[q_i+p_iG_{W_j}(s)\right]^{n_{ij}}.
\end{equation}
L'évaluation en $0$ donne directement 
$$
p_{X_{\text{ind}}}(0) = \prod_{i = 1}^{a}(q_i)^{n_i}. 
$$
En prenant le log et en dérivant dans l'équation \eqref{eq:fgp_X_ind}, il vient 
$$
G^{(1)}_{X_{\text{Ind}}}(s) = \sum_{i = 1}^{a}\sum_{i = 1}^{b}n_{ij}\frac{p_iG^{(1)}_{W_j}(s)}{q_i+p_iG_{W_j}(s)}G_{X_{\text{Ind}}}(s):=\sum_{i = 1}^{a}\sum_{i = 1}^{b}n_{ij}V_{ij}(s)
$$
or on sait que par définition de la fonction génératrice des probabilités, 
$$
G_{X_{\text{Ind}}}(s) = \sum_{x = 0}^mp_{X_{\text{Ind}}}(x)s^x,\text{ et }G^{(1)}_{X_{\text{Ind}}}(s) = \sum_{x = 1}^{m}xp_{X_{\text{Ind}}}(x)s^{x-1}.
$$
Cette remarque permet d'identifier les termes dans chacune des sommes définissant $G^{(1)}_{X_{\text{Ind}}}(s)$ avec 
$$
xp_{X_{\text{Ind}}}(x) = \frac{G^{(x)}_{X_{\text{Ind}}}(0)}{(x-1)!} = \sum_{i = 1}^{a}\sum_{i = 1}^{b}n_{ij}\frac{V^{(x-1)}_{ij}(0)}{(x-1)!},\text{ avec }x=1,\ldots, m.
$$
On écrit ensuite 
$$
V_{ij}(s) = \frac{p_iG^{(1)}_{W_j}(s)}{q_i+p_iG_{W_j}(s)}G_{X_{\text{Ind}}}(s)\Leftrightarrow V_{ij}(s) = \frac{p_i}{q_i}\left(G^{(1)}_{W_j}(s)G_{X_{\text{Ind}}}(s)-G_{W_j}(s)V_{ij}(s)\right).
$$
On applique ensuite la formule de Leibniz pour édriver $V_{ij}(s)$ à l'ordre $x-1$ et évaluer en $0$ pour obtenir
\begin{eqnarray}
v_{ij}(x)&:=&\frac{V^{(x-1)}_{ij}(0)}{(x-1)!}=\frac{p_i}{q_i}\sum_{l = 0}^{x-1}\left[(l+1)p_{W_{j}}(l+1)p_{X_{\text{Ind}}}(x-1-l)-p_{W_{j}}(l)v_{ij}(x-l)\right]\nonumber\\
&=&\frac{p_i}{q_i}\sum_{l = 1}^{x}\left[lp_{W_{j}}(l)p_{X_{\text{Ind}}}(x-l)-p_{W_{j}}(l-1)v_{ij}(x-l+1)\right]
\end{eqnarray}
On note que $p_{W_{j}}(0)=0$ et on fixe $v_{ij}(0) = 0$ par convention ce qui permet d'écrire
$$
v_{ij}(x)=\frac{p_i}{q_i}\sum_{l = 1}^{x\land m_j}p_{W_{j}}(l)\left[lp_{X_{\text{Ind}}}(x-l)-v_{ij}(x-l)\right].
$$
\end{frame}
\begin{frame}[allowframebreaks]

On peut tout de même s'assurer de la cohérence de la formule sur un cas simple. Soit un portefeuille de contrats d'assurance décès avec versement d'un capital $1$ en cas de décès. La probabilité $p_i$ pour $i = 1,\ldots, a$ correspond à la probabilité de décès dans la classe $i$, qui peuvent s'interpréter comme des classes d'âge. De plus on a 
$$
\P(W_j = 1)=1,\text{ pour tout }j = 1, \ldots b.
$$ 
La probabilité $p_{X_{\text{ind}}}(0)$ correspond a aucun décès dans le portefeuille. La probabilité $p_{X_{\text{ind}}}(1)$ correspond à un unique décès. La probabilité d'un décès dans la classe $i$ est donnée par 
$$
\binom{n_i}{1}q_i^{n_i-1}p_i,\text{ pour }i = 1,\ldots, a.
$$  
soit la probabilité qu'une v.a. de loi binomial soit égale à 1. La probabilité qu'on ait un décès dans la classe $i$ et pas de décès dans aucune autre classe (évènement noté $A_i$) est donnée par 
$$
n_i q_i^{n_i-1}p_i\underset{k=1,\ldots, a}{\prod_{k\neq i}}q_k^{n_k},\text{ pour }i = 1,\ldots, a.
$$
On a enfin 
$$
p_{X_\text{Ind}}(1) = \P\left(\cup_{i=1}^{a}A_i\right) = \sum_{i=1}^{a} n_i q_i^{n_i-1}p_i\underset{k=1,\ldots, a}{\prod_{k\neq i}}q_k^{n_k}.
$$
En appliquant la formule, on retrouve aisément 
$$
v_{ij}(1) = \frac{p_i}{q_i}\sum_{l = 1}^{1\land 1}(lp_{W_i}(l)p_{X_{\text {Ind}}}(1-l)- p_{W_i}(l-1)v_{ij}(1-l))=\frac{p_i}{q_i}p_{X_{\text {Ind}}}(0)
$$
puis 
$$
p_{X_\text{Ind}}(1) = \sum_{i=1}^a\sum_{j = 1}^b n_{ij}\frac{p_i}{q_i}p_{X_{\text {Ind}}}(0) = \sum_{i=1}^a n_iq_i^{n_i-1}p_i\underset{k=1,\ldots, a}{\prod_{k\neq i}}q_k^{n_k}.
$$
\end{frame}
\begin{frame}
Si le portefeuille de contrats est parfaitement homogène, c'est à dire que
\begin{itemize}
    \item Les $I_1,\ldots,I_n$ sont \iid $(p_1=p_2=\ldots p_n=p)$
    \item Les $W_1,\ldots,W_n$ sont \iid et indépendants des $I_1,\ldots,I_n$ 
\end{itemize}
alors 
$$
X_{ind} = \sum_{i=1}^{N}W_i
$$
où $N\sim\text{Bin}(n,p)$. \\

Le modèle individuel devient alors un modèle collectif. 
\end{frame}
\begin{frame}[allowframebreaks]{Références bibliographiques}
Mes notes s'inspirent des notes de Stéphane Loisel \cite{Lo19}
\bibliographystyle{plain}
\bibliography{MCS_notes}
\end{frame}
\end{document}
