\documentclass[11pt, addpoints, answers]{exam}

\usepackage[utf8]{inputenc}
\usepackage[T1]{fontenc}
\usepackage[margin  = 1in]{geometry}
\usepackage{amsmath, amscd, amssymb, amsthm, verbatim}
\usepackage{mathabx}
\usepackage{setspace}
\usepackage{float}
\usepackage{color}
\usepackage{graphicx}
\usepackage[colorlinks=true]{hyperref}
\usepackage{tikz}

\usetikzlibrary{shapes,arrows}
%%%<
\usepackage{verbatim}
%%%>
\usetikzlibrary{automata,arrows,positioning,calc}

\usetikzlibrary{trees}

\shadedsolutions
\definecolor{SolutionColor}{RGB}{214,240,234}

\newcommand{\bbC}{{\mathbb C}}
\newcommand{\R}{\mathbb{R}}            % real numbers
\newcommand{\bbR}{{\mathbb R}}
\newcommand{\Z}{\mathbb{Z}}            % integers
\newcommand{\bbZ}{{\mathbb Z}}
\newcommand{\bx}{\mathbf x}            % boldface x
\newcommand{\by}{\mathbf y}            % boldface y
\newcommand{\bz}{\mathbf z}            % boldface z
\newcommand{\bn}{\mathbf n}            % boldface n
\newcommand{\br}{\mathbf r}            % boldface r
\newcommand{\bc}{\mathbf c}            % boldface c
\newcommand{\be}{\mathbf e}            % boldface e
\newcommand{\bE}{\mathbb E}            % blackboard E
\newcommand{\bP}{\mathbb P}            % blackboard P

\newcommand{\ve}{\varepsilon}          % varepsilon
\newcommand{\avg}[1]{\left< #1 \right>} % for average
%\renewcommand{\vec}[1]{\mathbf{#1}} % bold vectors
\newcommand{\grad}{\nabla }
\newcommand{\lb}{\langle }
\newcommand{\rb}{\rangle }

\def\Bin{\operatorname{Bin}}
\def\Var{\operatorname{Var}}
\def\Geom{\operatorname{Geom}}
\def\Pois{\operatorname{Pois}}
\def\Exp{\operatorname{Exp}}
\newcommand{\Ber}{\operatorname{Ber}}
\def\Unif{\operatorname{Unif}}
\def\No{\operatorname{N}}
\newcommand{\E}{\mathbb E}            % blackboard E
\def\th{\theta }            % theta shortcut
\def\V{\operatorname{Var}}
\def\Var{\operatorname{Var}}
\def\Cov{\operatorname{Cov}}
\def\Corr{\operatorname{Corr}}
\newcommand{\epsi}{\varepsilon}            % epsilon shortcut

\providecommand{\norm}[1]{\left\lVert#1\right\rVert} %norm
\providecommand{\abs}[1]{\left \lvert#1\right \rvert} %absolute value

\DeclareMathOperator{\lcm}{lcm}
\newcommand{\ds}{\displaystyle}	% displaystyle shortcut

\def\semester{2019-2020}
\def\course{Modélisation Charge-Sinistre}
\def\title{\MakeUppercase{Examen de deuxième session}}
\def\name{Pierre-O Goffard}
%\def\name{Professor Wildman}

\setlength\parindent{0pt}

\cellwidth{.35in} %sets the minimum width of the blank cells to length
\gradetablestretch{2.5}

%\bracketedpoints
%\pointsinmargin
%\pointsinrightmargin

\begin{document}


\runningheader{\course  \vspace*{.25in}}{}{\title \vspace*{.25in}}
%\runningheadrule
\runningfooter{}{Page \thepage\ of \numpages}{}

% \firstpageheader{Name:\enspace\hbox to 2.5in{\hrulefill}\\  \vspace*{2em} Section: (circle one) TR: 3-3:50 \textbar\, TR: 5-5:50 \textbar\,  TR: 6-6:50(Xu) \textbar\,  TR: 6-6:50 }{}{Perm \#: \enspace\hbox to 1.5in{\hrulefill}\\ \vspace*{2em} Score:\enspace\hbox to .6in{\hrulefill} $/$\numpoints}
\extraheadheight{.25in}

\hrulefill

\vspace*{1em}

% Heading
{\center \textsc{\Large\title}\\
	\vspace*{1em}
	\course -- \semester\\
	Pierre-O Goffard\\
}
\vspace*{1em}

\hrulefill

\vspace*{2em}

\noindent {\bf\em Instructions:} On éteint et on range son téléphone.
\begin{itemize}
	\item La calculatrice et les appareils éléctroniques ne sont pas autorisés.
	\item Vous devez justifier vos réponses de manière claire et concise.
	\item Vous devez écrire de la manière la plus lisible possible. Souligner ou encadrer votre réponse finale.
	\item Document autorisé: Une feuille recto-verso manuscrite. 
\end{itemize}

\begin{center}
	\gradetable[h]
\end{center}

\smallskip

\section*{Une loi binomiale mélange beta}
\begin{questions}
\question Sur une période d'exercice donnée, le nombre de sinistres est modélisé par une variable aléatoire de loi $N\sim\text{Binomial}(n,\Theta)$ où $\Theta\sim\text{Beta}(a,b)$. On rappelle que la loi de probabilité d'une variable aléatoire N de loi binomiale $\text{Binomial}(n,p)$ est donnée par 
$$
\mathbb{P}(N = k)=\binom{n}{k}p^k(1-p)^{n-k},\text{ }k=0,\ldots, n.
$$
On rappelle que la loi de probabilité d'une variable aléatoire $\Theta\sim\text{Beta}(a,b)$ admet une densité 
$$
f_{\Theta}(\theta)=\frac{\theta^{a-1}(1-\theta)^{b-1}}{B(a,b)}\mathbb{I}_{[0,1]}(\theta),
$$
par rapport à la mesure de Lebesgue. La fonction beta  est définie par 
$$
B(a,b)=\int_{0}^{1}u^{a-1}(1-u)^{b-1}\text{d}u
.$$
\begin{parts}
\part Donner la loi de probabilité de $N$ en fonction de la fonction beta.
\begin{solution}
\begin{eqnarray*}
\mathbb{P}(N=k) &=& \int_{0}^1 \binom{n}{k}\theta^k(1-\theta)^{n-k}\frac{\theta^{a-1}(1-\theta)^{b-1}}{B(a,b)}\text{d}\theta\\
&=&\binom{n}{k}\frac{B(k+a,n-k+b)}{B(a,b)}
\end{eqnarray*}
pour $k = 0,\ldots, n$.
\end{solution}
\part Donner une expression de $\mathbb{E}(N)$ en fonction de $a,b$, et $n$.\\
\underline{Indications:} La fonction beta s'exprime en fonction de la fonction gamma de la façon suivante 
$$
B(a,b)=\frac{\Gamma(a)\Gamma(b)}{\Gamma(a+b)}
$$
On rappelle que la fonction gamma est définie par 
$$
\Gamma(z)=\int_{0}^{+\infty}e^{-t}t^{z-1}\text{d}t,\text{ }z>0.
$$
et vérifie en particulier $\Gamma(z+1)=z\Gamma(z)$.
\begin{solution}
$\mathbb{E}(N) = \mathbb{E}\mathbb{E}(N|\Theta))=n\mathbb{E}(\Theta)=n\frac{a}{a+b}$
\end{solution}
\part Calculer la variance $V(N)$.
\begin{solution}
\begin{eqnarray*}
V(N)&=&E(V(N|\Theta))+V(E(N|\Theta))\\
&=&nE(\Theta(1-\Theta))+n^2V(\Theta)\\
&=&n\frac{ab}{(a+b+1)(a+b)}+n^2\frac{ab}{(a+b+1)(a+b)^2}\\
&=&\frac{nab(a+b+n)}{(a+b+1)(a+b)^2}.
\end{eqnarray*}
\end{solution}
\part Donner une méthode d'estimation pour les paramètres de $N$, en supposant que le paramètre $n$ soit connu. Détailler la mise en oeuvre.  
\begin{solution}
Comme on vient de calculer la moyenne et la variance, on propose d'estimer $a$ et $b$ via la méthode des moments. On obtient 
$$
\widehat{a}= \frac{\bar N(n-\bar N)- S_N^2}{n(S_N^2/\bar N-1)+\bar N} ,\widehat{b} = \frac{(\bar N(n-\bar N)- S_N^2)(n-\bar N)}{\bar N(n(S_N^2/\bar X-1)+\bar N)}
$$
où $\bar N$ et $S_{N}^2$ désigne la moyenne et la variance empirique respectivement.
\end{solution}
\part La loi de $N$ appartient-elle à la famille de Panjer?
\begin{solution}
Non, il ne s'agit ni de la loi binomial, binomial négative ou Poisson.
\end{solution}
\end{parts}

\question Sur une période d'exercice donnée, la charge totale de sinistres est modélisée via un modèle collectif 
$$
X = \sum_{k=1}^N U_k,
$$
où le nombre de sinistres est une variable aléatoire de comptage de fonction de masse donnée par 
$$
\mathbb{P}(N=k)=p(1-p)^{k-1},\text{ }k\geq1
$$
et les montants de sinistres sont iid de loi exponentielle de densité de probabilité
$$
f_U(x) = \frac{e^{-x/\beta}}{\beta}\mathbb{I}_{(0,+\infty)}(x).
$$
 \begin{parts}
\part[2] Donner la moyenne et la variance de $X$.
\begin{solution}
$$
\mathbb{E}(X)=\mathbb{E}(N)\mathbb{E}(U)=\frac{\beta}{p}. 
$$
et 
$$
\mathbb{V}(X)=\mathbb{E}(N)\mathbb{V}(U)+ \mathbb{V}(N)\mathbb{E}(U)^2=\frac{\beta^2}{p^2}. 
$$
\end{solution}
\part[1] Donner la fonction génératrice des moments de $X$. Vous devez détailler les calculs.
\begin{solution}
$$
M_X(s) = \mathbb{E}(e^{sX})=G_N(M_U(s))=\frac{1}{1-\frac{\beta s}{p}}
$$
\end{solution}
\part[2] Donner la loi de probabilité de $X$. 
\begin{solution}
Il s'agit d'une loi exponentielle de paramètre $\beta/p$ par identification avec la fonction génératrice des moments calculer à la question précédente. Si on ne remarque pas ça, alors la densité de $X$ s'écrit 
$$
f_X(x)=\sum_{k=1}^{+\infty}p(1-p)^{k-1}f_U^{\ast k}(x) = \sum_{k=1}^{+\infty}p(1-p)^{k-1}\frac{e^{-x/\beta}x^{k-1}}{\beta^k k!}=\frac{p}{\beta}e^{-xp/\beta},
$$
ou $f_U^{\ast k}$ est le produit de convolution de rang $k$ de $f_U$ avec elle-même, il s'agit donc de la densité de la loi gamma de paramètres $k$ et $\beta$.
\end{solution}
\part[1] Supposons que nous n'observons que les charges totales de sinistres $x_1,\ldots,x_t$ et les nombres de sinistres $n_1,\ldots, n_t$ associées à $t$ périodes d'exercice. Proposer une méthode d'estimation des paramètres $p$ et $\beta$ basée sur les observations $x_1,\ldots, x_t$ et $n_1,\ldots, n_t$. Expliquer la mise en oeuvre, donner l'expression des estimateurs si possible. 
\begin{solution}
Le plus simple est de commencer par estimer $p$ en utilisant les nombres de sinistre $n_1,\ldots, n_t$, avec la méthode des moment, il vient $\widehat{p} = \left(\frac{1}{t}\sum_{s=1}^{t}n_s\right)^{-1}$, puis estimer $\beta$ via la méthode des moments avec $\widehat{\beta}=\widehat{p}\frac{1}{t}\sum_{s=1}^{t}x_s$.
\end{solution}
\end{parts}
\question Les montants de sinistres sont modéliser par une loi gamma $\text{Gamma}(k,m)$, $k,m>0$ de densité de probabilité 
$$
f_{\text{Gam}}(x)=\frac{e^{-x/m}x^{k-1}}{m^k\Gamma(k)}\mathbb{I}_{(0,+\infty)}(x),
$$
On rappelle que la fonction gamma est définie par 
$$
\Gamma(z)=\int_{0}^{+\infty}e^{-x}x^{z-1}\text{d}x.
$$
et vérifie en particulier $\Gamma(z+1) = z\Gamma(z)$. 
\begin{parts}
\part[1] Donner l'espérance et la variance des montants de sinistres en fonction de $k$ et $m$. 
\begin{solution}
$\mathbb{E}(U) = km\text{ et }\mathbb{V}(U) = km^2$
\end{solution}
\part[1] Donner la fonction génératrice des moments de la loi des sinistres.
\begin{solution}
$M_U(s) = \left(\frac{1}{1-ms}\right)^k$
\end{solution}
\part[2] Proposer une méthode d'estimation pour les paramètres $k$ et $m$. Donner l'expression des estimateurs pour un échantillon de montant de sinistres $u_1,\ldots, u_n$. 
\begin{solution}
On utilise la méthode des moments 
$$
\widehat{k}=\frac{\bar{U}^2}{S_U^2},\text{ }\widehat{m}=\frac{S_U^2}{\bar{U}}
$$
où $\bar U$ et $S_U^2$ désigne la moyenne et la variance empirique de l'échantillon de montants de sisnitres 
\end{solution}
\part[2] Pour affiner la modéliser du montant des sinistres, on associe à la loi gamma une loi de Pareto de densité
$$
f_{\text{Par}}(x) =\begin{cases}
\frac{\theta^\alpha\alpha}{x^{\alpha+1}},&x>\theta,\\
0,&\text{sinon.}
\end{cases} 
$$
pour modéliser les sinistres plus conséquents dans le cadre d'un modèle composite. Les conditions de régularité de la densité $f$ du modèle composite impose de fixer la valeur de certains paramètres. Donner l'expression du paramètre $m$ (de la loi gamma) et du paramètre de mélange $r$ du modèle composite en fonction des autres paramètres $k,\alpha $et $\theta$, de la fonction de répartition $F_{\text{Gam}}$ d'une loi gamma de paramètres $k$ et $m$ et la fonction gamma $\Gamma$.
\begin{solution}
La densité du modèle composite Gamma-Pareto est donnée par 
$$
f(x) = 
\begin{cases}
r\frac{f_{\text{gam}}(x)}{F_{\text{Gam}}(\theta)}&x\leq\theta\\
(1-r)f_{\text{Par}}(x)&x>\theta
\end{cases}
$$
les conditions de régularité $f(\theta^-)=f(\theta^+)$ et $f'(\theta^-)=f'(\theta^+)$ entraine
$$
\frac{r}{1-r} = \frac{f_{\text{Par}}(\theta)F_{\text{Gam}}(\theta)}{f_{\text{Gam}}(\theta)}
$$
de plus $f'_{Gam}(x) = f_{\text{Gam}}(x)\left[\frac{k-1}{x}-\frac{1}{m}\right]$ et $f'_{Par}(x)=-\frac{\alpha+1}{x}f_{Par}(x)$ On en déduit que 
$$
m = \frac{\theta}{k+\alpha},\text{ }r = \frac{\alpha\Gamma(k) F_{\text{Gam}}(\theta)e^{k+\alpha}(k+\alpha)^{-k}}{1+\alpha\Gamma(k) F_{\text{Gam}}(\theta)e^{k+\alpha}(k+\alpha)^{-k}}.
$$
\end{solution}

\end{parts}

\end{questions}
%-------------------------------TABLE-------------------------------
\newpage
\hrule
\vspace*{.15in}
\begin{center}
  \large\MakeUppercase{Formulaire}
\end{center}
\vspace*{.15in}
\hrule
\vspace*{.25in}

\renewcommand\arraystretch{3.5}
\begin{table}[H]
\begin{center}
\footnotesize
\begin{tabular}{|c|c|c|c|c|c|}

\hline
Nom & abbrev. & Loi & $\E(X)$ & $\Var(X)$ & FGM\\
\hline\hline
Binomiale & $\Bin(n,p)$ & $\binom{n}{k}p^k(1-p)^{n-k}$ pour $k = 0,\ldots,n$ & $np$ & $np(1-p)$ & $[(1-p)+pe^t]^n$\\
\hline
Poisson & $\Pois(\lambda)$ & $e^{-\lambda}\dfrac{\lambda^k}{k!}$ pour $k = 0,1,2,\ldots$ & $\lambda$ & $\lambda$ &$ \exp(\lambda(e^t-1))$\\
\hline
Binomiale Négative & $\mathcal{NB}(\alpha, p)$ & $\frac{\Gamma(\alpha+k)}{\Gamma(k+1)\Gamma(a)}(1-p)^{\alpha}p^k$ & $\dfrac{\alpha p }{1-p}$ & $\dfrac{\alpha p }{(1-p)^2}$ & $\left(\frac{1-p}{1-pe^t}\right)^\alpha$ pour  $t<-\ln(p)$\\
\hline
Uniforme & $\Unif(a,b)$ & $\begin{cases} \dfrac{1}{b-a} & a\leq t\leq b\\ 0 & \text{sinon}\end{cases}
$ & $\dfrac{a+b}{2}$ & $\dfrac{(b-a)^2}{12}$ & $\frac{e^{tb}-e^{ta}}{t(b-a)}$\\
\hline
Exponentielle & $\Exp(\lambda)$ & $\begin{cases} \lambda e^{-\lambda t} & t\geq 0 \\ 0 & t<0\end{cases}$ & $\dfrac{1}{\lambda}$ & $\dfrac{1}{\lambda^2}$ & $\frac{\lambda}{\lambda -t}$ pour $t<\lambda$\\
\hline
Gamma & $\text{Gamma}(\alpha,\beta)$ & $\begin{cases} \frac{\beta^\alpha t^{\alpha-1} e^{-\beta t}}{\Gamma(\alpha)} & t\geq 0 \\ 0 & t<0\end{cases}$ & $\dfrac{\alpha}{\beta}$ & $\dfrac{\alpha}{\beta^2}$ & $\left(\frac{\beta}{\beta -t}\right)^\alpha$ pour $t<\beta$\\
\hline
Normale & $\No(\mu,\sigma^2)$ & $\left(\dfrac{1}{\sqrt{2\pi\sigma^2}}\right)\operatorname{exp}{\left(\dfrac{-(t-\mu)^2}{2\sigma^2}\right)}$ & $\mu$ & $\sigma^2$ & $e^{\mu t}e^{\sigma^2t^2/2}$\\
\hline
\end{tabular}
\end{center}
\end{table}%
On rappelle la définition de la fonction Gamma avec 
$$
\Gamma(z) = \int_{0}^{+\infty}x^{z-1}e^{-x}\text{d}x.
$$
On note que pour $z\in \mathbb{N}$ alors $\Gamma(z) = (z-1)!$.
\end{document}