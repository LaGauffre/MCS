\documentclass[11pt,addpoints,answers]{exam}

\usepackage[utf8]{inputenc}
\usepackage[T1]{fontenc}
\usepackage[margin  = 1in]{geometry}
\usepackage{amsmath, amscd, amssymb, amsthm, verbatim}
\usepackage{mathabx}
\usepackage{setspace}
\usepackage{float}
\usepackage{color}
\usepackage{graphicx}
\usepackage[colorlinks=true]{hyperref}
\usepackage{tikz}

\usetikzlibrary{shapes,arrows}
%%%<
\usepackage{verbatim}
%%%>
\usetikzlibrary{automata,arrows,positioning,calc}

\usetikzlibrary{trees}

\shadedsolutions
\definecolor{SolutionColor}{RGB}{214,240,234} 

\newcommand{\bbC}{{\mathbb C}}
\newcommand{\R}{\mathbb{R}}            % real numbers
\newcommand{\bbR}{{\mathbb R}}
\newcommand{\Z}{\mathbb{Z}}            % integers
\newcommand{\bbZ}{{\mathbb Z}}
\newcommand{\bx}{\mathbf x}            % boldface x
\newcommand{\by}{\mathbf y}            % boldface y
\newcommand{\bz}{\mathbf z}            % boldface z
\newcommand{\bn}{\mathbf n}            % boldface n
\newcommand{\br}{\mathbf r}            % boldface r
\newcommand{\bc}{\mathbf c}            % boldface c
\newcommand{\be}{\mathbf e}            % boldface e
\newcommand{\bE}{\mathbb E}            % blackboard E
\newcommand{\bP}{\mathbb P}            % blackboard P

\newcommand{\ve}{\varepsilon}          % varepsilon
\newcommand{\avg}[1]{\left< #1 \right>} % for average
%\renewcommand{\vec}[1]{\mathbf{#1}} % bold vectors
\newcommand{\grad}{\nabla }
\newcommand{\lb}{\langle }
\newcommand{\rb}{\rangle }

\def\Bin{\operatorname{Bin}}
\def\Var{\operatorname{Var}}
\def\Geom{\operatorname{Geom}}
\def\Pois{\operatorname{Pois}}
\def\Exp{\operatorname{Exp}}
\newcommand{\Ber}{\operatorname{Ber}}
\def\Unif{\operatorname{Unif}}
\def\No{\operatorname{N}}
\newcommand{\E}{\mathbb E}            % blackboard E
\def\th{\theta }            % theta shortcut
\def\V{\operatorname{Var}}
\def\Var{\operatorname{Var}}
\def\Cov{\operatorname{Cov}}
\def\Corr{\operatorname{Corr}}
\newcommand{\epsi}{\varepsilon}            % epsilon shortcut

\providecommand{\norm}[1]{\left\lVert#1\right\rVert} %norm
\providecommand{\abs}[1]{\left \lvert#1\right \rvert} %absolute value

\DeclareMathOperator{\lcm}{lcm}
\newcommand{\ds}{\displaystyle}	% displaystyle shortcut

\def\semester{2020-2021}
\def\course{Modélisation Charge Sinistre}
\def\title{\MakeUppercase{Examen de deuxième session}}
\def\name{Pierre-O Goffard}
%\def\name{Professor Wildman}

\setlength\parindent{0pt}

\cellwidth{.35in} %sets the minimum width of the blank cells to length
\gradetablestretch{2.5}

%\bracketedpoints
%\pointsinmargin
%\pointsinrightmargin

\begin{document}


\runningheader{\course  \vspace*{.25in}}{}{\title \vspace*{.25in}}
%\runningheadrule
\runningfooter{}{Page \thepage\ of \numpages}{}

% \firstpageheader{Name:\enspace\hbox to 2.5in{\hrulefill}\\  \vspace*{2em} Section: (circle one) TR: 3-3:50 \textbar\, TR: 5-5:50 \textbar\,  TR: 6-6:50(Xu) \textbar\,  TR: 6-6:50 }{}{Perm \#: \enspace\hbox to 1.5in{\hrulefill}\\ \vspace*{2em} Score:\enspace\hbox to .6in{\hrulefill} $/$\numpoints}
\extraheadheight{.25in}

\hrulefill

\vspace*{1em}

% Heading
{\center \textsc{\Large\title}\\
	\vspace*{1em}
	\course -- \semester\\
	Pierre-O Goffard\\
}
\vspace*{1em}

\hrulefill

\vspace*{2em}

\noindent {\bf\em Instructions:} On éteint et on range son téléphone.
\begin{itemize}
	\item La calculatrice et les appareils éléctroniques ne sont pas autorisés.
	\item Vous devez justifier vos réponses de manière claire et concise.
	\item Vous devez écrire de la manière la plus lisible possible. Souligner ou encadrer votre réponse finale.
	\item \underline{Document autorisé:} Une feuille manuscrite recto-verso
\end{itemize}

\begin{center}
	\gradetable[h]
\end{center}

\smallskip

N'hésitez pas à utiliser le résultat des questions précédentes pour répondre à la question courante.
\begin{questions}
\question \textbf{Etude de la loi semi-gaussienne}\\

Le montant des sinistres est distribué comme une variable aléatoire continue et positive de loi semi-gaussienne $U\sim \text{SN}(\sigma)$ définie comme $U = |X|$ (valeur absolue de $X$) avec $X$ de loi normale centrée $X\sim\text{N}(0,\sigma^2)$.
\begin{parts}
\part[2] Montrer que la densité de $U$ est donnée par  
$$
f_U(x)=
\begin{cases}
\frac{\sqrt{2}}{\sigma\sqrt{\pi}}\exp\left(-\frac{x^2}{2\sigma^2}\right),&x>0,\\
0,&\text{ Sinon}.
\end{cases}
$$
\begin{solution}
On observe que 
$$
F_U(x) = \mathbb{P}(-x\leq X\leq x) = F_X(x) - F_X(-x)
$$
puis on dérive par rapport à $x$.
\end{solution}
\part[2] Calculer l'espérance et la variance de $U$ (détailler les calculs)
\begin{solution}
$\mathbb{E}(U)=\frac{\sigma\sqrt{2}}{\sqrt{\pi}}$, et  $\mathbb{V}(U)=\sigma^2\left(1-\frac{2}{\pi}\right)$. 
\end{solution}
\end{parts}
\question[2] \textbf{Estimation paramétrique de la loi semi-gaussienne}\\

On souhaite calibrer un modèle semi gaussien $\text{IG}(\sigma)$ à notre historique de données
$$(u_1,\ldots, u_n).$$ 
Donner l'expression d'un estimateur $\widehat{\sigma}$ de $\sigma$ par la méthode des moments. Quelle est la loi de cette estimateur lorsque le nombre d'observation est grand et pourquoi?
\begin{solution}
 Méthode des moments 
	$$
\widehat{\sigma} = \frac{\sqrt{\pi}\sum_{i = 1}^n u_i}{\sqrt{2}n}
	$$
L'estimateur suit une loi normale lorsque la taille de l'échantillon grandit, par application du théorème centrale limite.

\end{solution}

\question[2] \textbf{Pros and cons de la distribution semi gaussienne}\\
Quels sont selon vous les inconvénients/avantages de cette loi semi Gaussienne pour modéliser des montants de sinistres?
\begin{solution}
Avantage
\begin{itemize}
	\item Stabilité par convolution, une somme de variables aléatoires de loi demi gaussienne suit une loi demi gaussienne
	\item Estimation simple via la méthode des moments, de plus la loi de l'estimateur est connu ce qui est pratique pour la construction d'intervalle de confiance.  
\end{itemize}
Inconvénient
\begin{itemize}
	\item La loi demi gaussienne est une distribution à queue légère, ce qui peut être génant pour la modélisation des sinistres de forte intensité.
\end{itemize}
\end{solution}
\question \textbf{Modèle Poisson mélange pour la loi de la fréquence des sinistres}\\
La fréquence des sinistres $N$ suit une loi Poisson zéro-inflatée $N\sim\text{ZI-Pois}(p,\lambda)$ définie par
$$
N = I\cdot M,
$$
où $I\sim\text{Bin}(1, p)$ et $M\sim\text{Pois}(\lambda)$ sont deux variables aléatoires indépendantes. 
\begin{parts}
\part[2] Donner la moyenne et la variance de $N$ en justifiant
\begin{solution}
La moyenne est donnée par
$$
\E(N) = \E(I\cdot M) = \E(I)\E(M) = p\cdot\lambda 
$$
et la variance par 
$$
\V(N) = \V(\E(I\cdot M| I ) + \E(\V(I\cdot M| I )) = \V(I)\E(M)^2 + \E(I) \V(M) = p(1-p)\lambda^2 + p\lambda     
$$

\end{solution}
\part[2] Donner la fonction génératrice des probabilités de $N$ définie par 
$$
G_N(s)=\mathbb{E}(s^N),\text{ }s\in\mathbb{R}.
$$
\begin{solution}
$$
G_N(s) =  1-p + p\exp[\lambda(s-1)]
$$
\end{solution}
\part[1] Quel est l'intérêt d'un tel modèle par rapport à une loi de Poisson simple?
\begin{solution}
Plus de paramètres donc plus de flexibilité que la loi de Poisson, plus grande occurence de zéros.
\end{solution}
\end{parts}
\question \textbf{Modèle collectif avec des montants de sinistres de loi semi gaussienne et une fréquence de sinistre de loi Poisson mélange}\\

La charge totale associée à un portefeuille de contrats d'assurance non vie est modèlisé à l'aide d'un modèle collectif 
$$
X = \sum_{i = 1}^{N}U_i.
$$
pour lequel le nombre de sinistre suit une loi Poisson zéro-inflatée $\text{ZI-Pois}(p, \lambda)$ et le montant des sinsitres par une loi semi gaussienne $U\sim\text{SN}(\sigma)$. 

\begin{parts}
\part[2] Calculer la moyenne et la variance de $X$ en fonction de $\sigma, p$ et $\lambda$.
\begin{solution}
On applique les formules du cours 
$$
\mathbb{E}(X_{\text{col}}) =\mathbb{E}(N)\mathbb{E}(U) = p\lambda\frac{\sqrt{2}}{\sqrt{\pi}}\sigma, 
$$
et
$$
\mathbb{V}(X_{\text{col}}) = \mathbb{E}(N)\mathbb{V}(U) + \mathbb{V}(N)\mathbb{E}(U)^2 = p\lambda\sigma\left(1-\frac{2}{\pi}\right) + ( p(1-p)\lambda^2 + p\lambda  )\frac{2}{\pi}\sigma^2
$$
\end{solution}
\part[2] Donner l'approximation normale de la fonction de répartition de $X$ en fonction de $\sigma$, $p, \lambda$ et $\phi$ la fonction de répartition de la loi normale $\text{N}(0,1)$.
\begin{solution}
$$
F_X(x)\approx \phi\left(\frac{x- p\lambda\frac{\sqrt{2}}{\sqrt{\pi}}\sigma }{p\lambda\sigma\left(1-\frac{2}{\pi}\right) + ( p(1-p)\lambda^2 + p\lambda  )\frac{2}{\pi}\sigma^2}\right) 	
$$
\end{solution}
\part[2] Cette méthode d'approximation est elle adaptée? Proposer une méthode d'approximation alternative et détailler sa mise en oeuvre. 
\begin{solution}
Non, car la loi de la fréquence des sinistres n'est pas Poisson, on peut utiliser l'approximation gamma ou la méthode FFT (pas l'algorithme de Panjer)
\end{solution}

\end{parts}

\end{questions}
%-------------------------------TABLE-------------------------------
\newpage
\hrule
\vspace*{.15in}
\begin{center}
  \large\MakeUppercase{Formulaire}
\end{center}
\vspace*{.15in}
\hrule
\vspace*{.25in}

\renewcommand\arraystretch{3.5}
\begin{table}[H]
\begin{center}
\footnotesize
\begin{tabular}{|c|c|c|c|c|c|}

\hline
Nom & abbrev. & Loi & $\E(X)$ & $\Var(X)$ & FGM\\
\hline\hline
Binomial & $\Bin(n,p)$ & $\binom{n}{k}p^k(1-p)^{n-k}$ & $np$ & $np(1-p)$ & $[(1-p)+pe^t]^n$\\
\hline
Poisson & $\Pois(\lambda)$ & $e^{-\lambda}\dfrac{\lambda^k}{k!}$ & $\lambda$ & $\lambda$ &$ \exp(\lambda(e^t-1))$\\
\hline
Geometric & $\Geom(p)$ & $(1-p)^{k-1}p$ & $\dfrac{1}{p}$ & $\dfrac{1-p}{p^2}$ & $\frac{pe^t}{1-(1-p)e^t}$ pour  $t<-\ln(1-p)$\\
\hline
Uniform & $\Unif(a,b)$ & $\begin{cases} \dfrac{1}{b-a} & a\leq t\leq b\\ 0 & \text{sinon}\end{cases}
$ & $\dfrac{a+b}{2}$ & $\dfrac{(b-a)^2}{12}$ & $\frac{e^{tb}-e^{ta}}{t(b-a)}$\\
\hline
Exponential & $\Exp(\lambda)$ & $\begin{cases} \lambda e^{-\lambda t} & t\geq 0 \\ 0 & t<0\end{cases}$ & $\dfrac{1}{\lambda}$ & $\dfrac{1}{\lambda^2}$ & $\frac{\lambda}{\lambda -t}$ pour $t<\lambda$\\
\hline
Normal & $\No(\mu,\sigma^2)$ & $\left(\dfrac{1}{\sqrt{2\pi\sigma^2}}\right)\operatorname{exp}{\left(\dfrac{-(t-\mu)^2}{2\sigma^2}\right)}$ & $\mu$ & $\sigma^2$ & $e^{\mu t}e^{\sigma^2t^2/2}$\\
\hline
\end{tabular}
\end{center}
\end{table}%
\bibliographystyle{plain}
\bibliography{IG_distribution}
\end{document}